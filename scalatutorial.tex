\documentclass[10pt,a4paper]{book}
\usepackage[utf8]{inputenc}
\usepackage{amsmath}
\usepackage{amsfonts}
\usepackage{amssymb}
\usepackage{hyperref}
\usepackage{bookmark}
\usepackage{marginnote}
\usepackage[left=2cm,right=2cm,top=2cm,bottom=2cm]{geometry}
\usepackage{paralist}
\usepackage{graphicx}
\usepackage{caption}
\usepackage{bbding}
\usepackage{keystroke}
\usepackage{subcaption}
\usepackage{hyperref}
\let\checkmark\relax
\usepackage{dingbat} 
\newcommand{\HRule}{\rule{\linewidth}{0.5mm}}
\renewcommand*\chaptername{Bölüm}
\renewcommand*\contentsname{İçindekiler}
\begin{document}

\begin{titlepage}
\begin{center}
\HRule \\[1.5cm]
{ \huge \bfseries A Scala Tutorial for Java programmers}\\[1.5cm]
\HRule \\[1.5cm]

\begin{minipage}{0.4\textwidth}
\begin{flushleft} \large
\emph{Yazarlar:}\\
Michel \textsc{Schinz}\\
Philipp \textsc{Haller}
\end{flushleft}
\end{minipage}
\begin{minipage}{0.4\textwidth}
\begin{flushright} \large
\emph{Çeviren:} \\
Mert \textsc{Eker}\\

\end{flushright}
\end{minipage}
\vfill
{\large \today}

\end{center}
\end{titlepage}
\tableofcontents

\begin{chapter}{Giriş}
This document gives a quick introduction to the Scala language and compiler. It is intended for people who already have some programming experience and want an overview of what they can do with Scala. A basic knowledge of object-oriented programming, especially in Java, is assumed.
\end{chapter}

\begin{chapter}{İlk Örnek}


\begin{verbatim}
   object HelloWorld {
      def main(args: Array[String]) {
          println("Hello, world!")
      }
   }
\end{verbatim}



\begin{section}{Compiling the example}

\end{section}

\begin{section}{Running the example}

\end{section}

\end{chapter}

\begin{chapter}{Interaction with Java}

\end{chapter}

\begin{chapter}{Everything is an object}

\begin{section}{Numbers are objects}

\end{section}

\begin{section}{Functions are objects}

\begin{subsection}{Anonymous functions}

\end{subsection}

\end{section}

\end{chapter}

\begin{chapter}{Sınıflar}

\begin{section}{Methods without arguments}

\end{section}

\begin{section}{Inheritance and overriding}

\end{section}

\end{chapter}

\begin{chapter}{Case classes and pattern matching}

\end{chapter}

\begin{chapter}{Traits}

\end{chapter}

\begin{chapter}{Genericity}

\end{chapter}

\begin{chapter}{Interaction with Java}

\end{chapter}
\end{document}